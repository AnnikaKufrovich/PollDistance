\documentclass[12pt]{article}

\usepackage{fullpage} %full page typesetting
\usepackage{setspace} %allows for non-singlespacing
\usepackage{graphicx} %graphics capabilities
\usepackage{latexsym} %extra symbols
\usepackage{rotating} %rotation for figures
\usepackage{longtable} %tables that fill more than a single page
\usepackage{hyperref} %hypertext links in the document
\usepackage{natbib} %better bibliographies
\usepackage{authblk} %author and affiliation in opening
\usepackage{mathpazo} %use palatino font, rather than times

\title{\tb{Polling Location Notes}}

\author{Annika}
\affil{New College of Florida}

\date{June 6, 2020}
\doublespacing

\begin{document}

\maketitle
\clearpage

\tableofcontents
\clearpage

\section{Note on Updates}
\begin{itemize}
  \item Previous iterations of these notes had links + notes for each individual county. That section has been paired down since the web-scraped links can be easily reconstructed from the web-scraper rmd file and. Notes are now diminished to look at what was found, what wasn't and where to go from here.
\end{itemize}


\section{Entire State}
\begin{itemize}
  \item As far as historical polling locations, I was not able to find dataset or source for polling locations across the state that is not current. I am working on getting these through public records request(s).
  \item \href{https://registration.elections.myflorida.com/CheckVoterStatus}{This find polling location page} for the state of Florida has a search function which navigates to individual SoE websites for polling location. Because of the non-uniformity in destinations it is not easily web-scraped and all data would only be current
  \item While also current, voterfocus.com has lists of current voting locations by precinct for each county in florida (linked to by most SOE sites) but you're redirected if you don't go to the specific county address (eg only going to voterfocus.com). I've used these links for the webscraper and only a couple counties are missed in for loop I wrote.
\end{itemize}

\clearpage

\section{Individual Counties}

Because most of the state is addressed through the voterfocus website used in the webscraper file this section is now lists of when precincts were last drawn for each county when data was last collected (8/3/2020). If we assume polling locations only change when precincts are redrawn this may be useful in determining historical polling locations in some counties


\subsection{Precincts drawn before 2016 or undated}

Bradford County
\begin{itemize}
  \item Precincts updated June 2014, based on map linked \href{https://www.bradfordelections.com/Portals/Bradford/Documents/precinct_map.pdf?ver=2014-06-25-110922-550}{here}
  \item Based on \href{https://www.bradfordelections.com/Portals/Bradford/Documents/Bradford%20Polling%20Locations%20with%20Address.pdf?ver=2019-09-26-161907-300}{this link} it seems there may have been a change to polling locations in September 2019. May want to reach out to SoE if looking farther back than 2019
\end{itemize}
Broward County
\begin{itemize}
  \item Did not find when precincts were last drawn or polling location history, assume polling locations are current unless told otherwise by SoE
\end{itemize}
Calhoun County
\begin{itemize}
  \item Precincts updated 2011, see \href{https://www.votecalhoun.com/portals/calhoun/documents/calhouncountyprecinctmap2011.pdf}{this link}.
\end{itemize}
Charlotte County
\begin{itemize}
  \item Any precinct changes not clear because the \href{https://www.charlottevotes.com/Maps/Precinct-Maps}{maps here} are undate. Assume current unless contacting SoE directly confirms otherwise.
\end{itemize}
Clay County
\begin{itemize}
  \item Any precinct changes not clear because the interactive \href{https://www.clayelections.gov/General-Information/Voter-Information-Map}{map here} is undated. Assume current unless contacting SoE directly confirms otherwise.
\end{itemize}
Collier County
\begin{itemize}
  \item Any precinct changes not clear because the interactive \href{https://www.colliervotes.com/Voting-System-Maps-Stats/Precinct-Map-Voting-Boundaries}{map here} is undated. Assume current unless contacting SoE directly confirms otherwise.
\end{itemize}
DeSoto County
\begin{itemize}
  \item No map that I could find or date of last precinct update. Assume current unless contacting SoE directly confirms otherwise.
\end{itemize}
Dixie County
\begin{itemize}
  \item Precincts updated 2012, see \href{https://www.dixievotes.com/Portals/Dixie/Documents/Maps/Precinct2012.pdf?ver=2014-06-04-143440-740}{map here}.
\end{itemize}
Escambia County
\begin{itemize}
  \item Precincts updated 2012, see \href{https://escambiavotes.com/docs/default-source/aws/Precincts_by_District_Combined.pdf}{map here}.
  \item Emergency polling location change for 2020 primary at \href{https://escambiavotes.com/news/2020/03/11/polling-location-change}{this link}
  \item May be able to look at polling locations that were not changed in an emergence historically
\end{itemize}
Franklin County
\begin{itemize}
  \item Because the interactive \href{https://www.arcgis.com/apps/webappviewer/index.html?id=02420ce7216f440a9b509e63510bfd9f}{map here} does not show when precincts were last drawn or polling location history assume polling locaitons are only current.
\end{itemize}
Gadsden County
\begin{itemize}
  \item Precincts updated 2013, see \href{https://www.voterfocus.com/PrecinctFinder/precinctDirectory?county=FL-GAD}{map here}.
\end{itemize}
Gilchrist County
\begin{itemize}
  \item Because the pdf \href{https://www.votegilchrist.com/Portals/Gilchrist/Documents/cm_voter__comm_dist.pdf}{map here} does not show when precincts were last drawn or polling location history, assume polling locations are only current.
\end{itemize}
Glades County
\begin{itemize}
  \item Because there are no maps or pages that show when precincts were last drawn or polling location history, assume polling locations are only current.
\end{itemize}
Gulf County
\begin{itemize}
  \item Because there are no maps or pages that show when precincts were last drawn or polling location history, assume polling locations are only current.
\end{itemize}
Hamilton County
\begin{itemize}
  \item Because there are no maps or pages that show when precincts were last drawn or polling location history, assume polling locations are only current.
\end{itemize}
Hardee County
\begin{itemize}
  \item Because there are no maps or pages that show when precincts were last drawn or polling location history, assume polling locations are only current.
\end{itemize}
Holmes County
\begin{itemize}
  \item Precincts updated 2014, see \href{https://www.holmeselections.com/Portals/Holmes/Documents/Precinct%20Map.pdf?ver=2014-02-07-162932-680}{map here}.
  \item It appears polling locations are same on map as the ones webscraped so this more certainly able to be used historically.
\end{itemize}
Indian River County
\begin{itemize}
  \item Because there are no maps or pages that show when precincts were last drawn or polling location history, assume polling locations are only current.
\end{itemize}
Jefferson County
\begin{itemize}
  \item Because there are no maps or pages that show when precincts were last drawn or polling location history, assume polling locations are only current.
\end{itemize}
Lafayette County
\begin{itemize}
  \item Because there are no maps or pages that show when precincts were last drawn or polling location history, assume polling locations are only current.
\end{itemize}
Lee County
\begin{itemize}
  \item Because the interactive map does not show when precincts were last drawn or polling location history, assume polling locations are only current.
\end{itemize}
Manatee County
\begin{itemize}
  \item Precincts updated 2013, see \href{https://www.votemanatee.com/portals/manatee/documents/pct2013.pdf}{map here}.
  \item It appears polling locations are same on map as were scraped so these are more certainly historical.
\end{itemize}

\clearpage

\subsection{Precincts drawn in 2016}

Alachua County
\begin{itemize}
  \item Precincts updated March or May 2016, see \href{https://www.votealachua.com/Elections/Maps}{this link}.
  \item Polling location changes for 2020 clearly stated \href{https://www.votealachua.com/Voters/Precinct-Voting}{here}, could use other precincts' polling locations historically
\end{itemize}
Pinellas County
\begin{itemize}
  \item Precincts updated April 2016, see \href{https://www.votepinellas.com/General-Information/Maps/District-Maps}{this link}.
\end{itemize}



\clearpage

\subsection{Precincts drawn in 2017}

Bay County
\begin{itemize}
  \item Precincts updated in July 2017 and special "Mega" polling locations for 2018, see maps at \href{https://www.bayvotes.org/Voter-Info/Maps-and-Boundaries}{this link}.
\end{itemize}
Hendry County
\begin{itemize}
  \item Precincts updated 2017, see map at \href{https://www.hendryelections.org/Portals/Hendry/2016-2018%20Precincts.pdf}{this link}.
\end{itemize}
Hillsborough County
\begin{itemize}
  \item It seens precincts updated June 2017, see maps at \href{https://www.votehillsborough.org/RESEARCH-DATA/MAPS-DISTRICTS-HBC-Indivudual-Maps}{this link}.
\end{itemize}
Jackson County
\begin{itemize}
  \item It seens precincts updated July 2017, see fine print of the precinct map at \href{https://www.jacksoncountysoe.org/Portals/Jackson/Documents/Accessibility%20PDF%20Fixes/Current_Precincts_07_18_2017FIXED.pdf?ver=2019-01-24-084548-780}{this link}.
  \item Seems like the polling locations on map match what was scraped so should be more certainly historical.
\end{itemize}


\clearpage

\subsection{Precincts drawn in 2018}

Baker County
\begin{itemize}
  \item Precincts updated 2018, see pdf map at \href{https://www.bakerelections.com/Portals/Baker/Documents/Precinct%20Map%202018.pdf}{this link} (date in title).
\end{itemize}
Flagler County
\begin{itemize}
  \item Precincts updated 2018, see \href{https://www.flaglerelections.com/For-Voters/District-Precinct-Maps}{this link}.
\end{itemize}
Highlands County
\begin{itemize}
  \item Precincts updated October 2018, see \href{https://www.votehighlands.com/Portals/Highlands/Documents/2018%20Precincts%20Map_092818.pdf?ver=2018-10-16-131000-593}{this link}.
\end{itemize}
Lake County
\begin{itemize}
  \item Precincts updated July 2018, see \href{https://www.lakecountyfl.gov/pdfs/gis/maps/VotingPrecincts_34x44-ADA.pdf}{this link}.
\end{itemize}
Liberty County
\begin{itemize}
  \item Precincts updated 2018, see \href{http://www.libertycad.com/wp-content/uploads/2018/11/2018-VP-MAP.pdf}{this link}.
\end{itemize}


\clearpage

\subsection{Precincts drawn in 2019}
Leon County
\begin{itemize}
  \item Precincts maps dated 2019 but pdf link has 2016, see \href{https://www.leonvotes.org/Portals/Leon/Documents/Maps_and%20Demographics/Pdfs/2016/Pct_Mapbook_2016.pdf}{this link}. Go with 2019 to be safe.
\end{itemize}

\clearpage

\subsection{Precincts drawn in 2020}

Brevard County
\begin{itemize}
  \item Many individual maps of precincts and polling locations, see \href{https://www.votebrevard.com/m/Maps/Precinct-Maps-Copy}{this link}.
  \item District maps show that there were updates as recently as 2020 \href{https://www.votebrevard.com/m/Maps/District-Maps}{available here}.
\end{itemize}
Citrus County
\begin{itemize}
  \item Some hitorical pdf maps of precincts but not polling locations, for 2020 precinct map see \href{https://www.votecitrus.com/Portals/Citrus/Pct%20map%20February%20Layout_2.pdf?ver=2020-04-28-200823-073}{this link}.
\end{itemize}
Columbia County
\begin{itemize}
  \item For 2020 precinct map see \href{https://www.votecolumbia.com/Portals/Columbia/Documents/SOE_Prec2019.pdf?ver=2020-06-29-135723-083}{this link}.
\end{itemize}
Dixie County
\begin{itemize}
  \item Maps have varying dates, see dates on \href{https://www.duvalelections.com/Voter-Information/Maps/Precinct-Maps}{maps here}.
\end{itemize}
Hernando County
\begin{itemize}
  \item 2020 precinct \href{https://www.hernandovotes.com/Portals/Hernando/Voting%20Precincts062520.pdf?ver=2020-06-26-132804-290}{maps here}.
\end{itemize}



\clearpage



\section{Current Recommendations}
Most of this data should be useable for the upcoming 2020 election (re-scrape the day before and day of just to be sure). It may be worthwhile to expand this webscraper to other states in the future. It would also be useful to compare these locations to ones obtained through public records requests to ensure they are accurate. It would also be worthwhile to keep a record of previous and current polling locations in some regularity going forward to create a database.
\end{document}
